% Created 2019-07-06 Sat 15:10
% Intended LaTeX compiler: pdflatex
\documentclass[11pt]{article}
\usepackage[utf8]{inputenc}
\usepackage[T1]{fontenc}
\usepackage{graphicx}
\usepackage{grffile}
\usepackage{longtable}
\usepackage{wrapfig}
\usepackage{rotating}
\usepackage[normalem]{ulem}
\usepackage{amsmath}
\usepackage{textcomp}
\usepackage{amssymb}
\usepackage{capt-of}
\usepackage{hyperref}
\usepackage{minted}
\author{Nicolás Luarte}
\date{\today}
\title{Francisca}
\hypersetup{
 pdfauthor={Nicolás Luarte},
 pdftitle={Francisca},
 pdfkeywords={},
 pdfsubject={},
 pdfcreator={Emacs 25.2.2 (Org mode 9.2.3)}, 
 pdflang={English}}
\begin{document}

\maketitle
\tableofcontents

\section{Ciclos}
\label{sec:orgef2f7b2}
\begin{center}
\begin{tabular}{lll}
Ejercicio & Protocolo & Bloque\\
\hline
Sentadilla de competencia & x5@8, 10\%, 5x5 & A\\
Banca de competencia & x6@8, 10\%, 5x6 & A\\
Press banca pies arriba (agarre cerrado) & x6@8, 15\%, 5x6 & A\\
\hline
Press banca agarre cerrado & x7@8, 10\%, 4x6 & B\\
Peso muerto convencional & x3@8, 10\%, 2x3 & B\\
Peso muerto sumo & x3@8, 10\%, 2x3 & B\\
\hline
\end{tabular}
\end{center}
\section{Comentarios técnicos}
\label{sec:org0f2516b}
\subsection{01/07/2019, ciclo 1, bloque a}
\label{sec:org13402bc}
\subsubsection{Sentadilla de competencia}
\label{sec:org23894c4}
\begin{enumerate}
\item Juntar un poquito más los pies
\item Reducir el ángulo de las punteras de los pies
\item Concentrarse en mantener las rodilas adelante
\item Reducir la velocidad de descenso
\item Pensar en llevar las rodillas hacia adelante
\item Realizar todo los sets luego del drop, con bandas alrededor de las
piernas, a altura de las rodillas
\end{enumerate}
\subsubsection{Banca de competencia}
\label{sec:orge08fe83}
\begin{enumerate}
\item Pausa activa, piensa que hay un huevo en tú pecho y no lo puedes
reventar
\item Retracción escapular, piensa que tienes que apretar una moneda que
te ponen en tú espalda con los hombros
\end{enumerate}
\subsubsection{Press banca pies arriba}
\label{sec:org6171e4a}
\begin{enumerate}
\item Mismos comentarios que para press banca
\end{enumerate}
\subsection{02/07/2019, ciclo 1, bloque b}
\label{sec:orgd81f322}
\subsubsection{Press banca agarre cerrado}
\label{sec:org941d3c4}
\begin{enumerate}
\item Centrarse en pensar que tienes un huevo en el pecho y por lo tanto
debes tener cuidado de no romperlo al bajar la barra
\item Baja un poco mas lento la barra
\end{enumerate}
\subsubsection{Peso muerto convencional}
\label{sec:orgcaeff0e}
\begin{enumerate}
\item Debes tomar mas aire por tú boca, llevarlo a tu estomago, inflarlo
y luego apretar los abdominales hacia abajo
\item El bloqueo debe ser súper agresivo y rápido, toma la costumbre
desde el primer calentamiento, bloquear caderas rápido
\end{enumerate}
\subsubsection{Peso muerto sumo}
\label{sec:orgccba87b}
\begin{enumerate}
\item Por ahora, solo trabajar en ir aumentando la apertura de las rodillas
\end{enumerate}
\subsection{04/07/2019, ciclo 1, bloque a}
\label{sec:org8870619}
\subsubsection{Sentadilla de competencia}
\label{sec:orgb995f1d}
\begin{enumerate}
\item Juntar un poquito más los pies
\item Reducir el ángulo de las punteras de los pies
\item Concentrarse en mantener las rodilas adelante
\item Reducir la velocidad de descenso
\item Pensar en llevar las rodillas hacia adelante
\item Realizar todo los sets luego del drop, con bandas alrededor de las
piernas, a altura de las rodillas
\item Bajar levemente los codos y hacer retracción escapular antes de
sacar la barra del rack, manteniendola durante todo el movimiento
\item Hacer mas exagerada la tomada de aire, y hacer también antes de
sacar la barra
\end{enumerate}
\subsubsection{Banca de competencia}
\label{sec:org3976495}
\begin{enumerate}
\item Pausa activa, piensa que hay un huevo en tú pecho y no lo puedes
reventar
\item Retracción escapular, piensa que tienes que apretar una moneda que
te ponen en tú espalda con los hombros
\item Evitar ese rebote que haces para sacar la barra del pecho, recuerda
que la barra apenas debe tocar el pecho y luego salir de ahí
\item Piensa es llevar el pecho hacia la barra, en vez de la barra hacia
el pecho, con eso busco que exageres sacar el pecho mientras haces
la banca
\end{enumerate}
\subsubsection{Press banca pies arriba}
\label{sec:orge99e216}
\begin{enumerate}
\item Mismos comentarios que para press banca
\end{enumerate}
\section{Registro de progreso}
\label{sec:org3bb6f42}
\subsection{Ciclo 1}
\label{sec:orge872ca2}
\begin{center}
\label{tab:org03c8200}
\begin{tabular}{lrrl}
Ejercicio & RPE & Peso & Fecha\\
\hline
Sentadilla de competencia & 8 & 65 & 01/07/2019\\
Press banca de competencia & 8 & 38 & 01/07/2019\\
Press banca pies arriba & 10 & 38 & 01/07/2019\\
Press banca agarre cerrado & 8 & 34 & 02/07/2019\\
Peso muerto convencional & 8 & 88 & 02/07/2019\\
Peso muerto sumo & 8 & 70 & 02/07/2019\\
Sentadilla de competencia & 8 & 65 & 04/07/2019\\
Press banca de competencia & 8 & 43 & 04/07/2019\\
Press banca pies arriba & 8 & 34 & 04/07/2019\\
 &  &  & \\
\end{tabular}
\end{center}
\subsection{Plots}
\label{sec:org408ba8b}
\subsubsection{Ciclo 1}
\label{sec:orgf6a1ae8}
\begin{center}
\includegraphics[width=.9\linewidth]{tmp.png}
\end{center}
\end{document}
